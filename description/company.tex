\resheading{Project experience and internship experience}
  \begin{itemize}[leftmargin=*]
    \item
      \ressubsingleline{2017 MCM/ICM}{Captain}{2017.02}
      {\small
      \begin{itemize}
      \item We use the analytic hierarchy process to carry out statistical analysis on the factors affecting \\the urban living environment, and at the same time analyze the factors affecting the urban \\living environment and the weight of the factors.
        \item We use fuzzy comprehensive evaluation and gray prediction methods to predict the changes of \\various indicators.
        \item We study urban sustainable development models, and after joint discussions within the group, \\we write the sustainable development papers to describe changes in various indicators.
      \end{itemize}
      }
    
    \item
      \ressubsingleline{The 6th TaiDi competition}{Captain}{2018.03 -- 2018.04}
      {\small
      \begin{itemize}
        \item Firstly,we cleaned the original data set and used regular expressions to extract the names of \\TV programs in the original data and categorized the programs. We also implicitly score\\ TV programs based on the user's viewing time and frequency.
        \item Secondly,we constructed a user label system table and a product label system table, and \\performed user portraits based on the time characteristics of user viewing information and we \\matched programs and program categories
to categorize the TV programs.
        \item  To better implement the recommendation, we use  user-base collaborative filtering, item-base \\collaborative filtering, SVD and hybrid Recommendation algorithm.The SVD algorithm\\ has the best effect.
		\item We use K-means method to package users and products, and recommend users without any \\ historical behavior, effectively solving the problems of user cold start and product cold start.
        \item After the competition, we optimized the original method and tried to use Text-cnn to recommend \\the original data , which improved the accuracy of the recommendation.
      \end{itemize}
      }

    \newpage
    \item
      \ressubsingleline{Douban web crawler data analysis}{Personal project}{2019.09 – 2019.12}
      {\small
      \begin{itemize}
       \item Project website:https://github.com/W55699/doubanbook − web − crawler
       \item I use regular expressions, and by creating a thread pool,\\ multi-threaded to crawl Douban books information
       \item I converted the information obtained by the crawler into csv file\\ and stored it in the mysql database.
       \item I use pandas to read csv, and do data visualization analysis and statistical analysis.
       \item I use pca to reduce the dimensionality of the data, extract key information, and then use the k-means algorithm to mine similar types of book information.
       \item I divided the data into training set and test set based on the information after pca's dimensionality reduction, combined with the label of the data, and divided the data into two categories, and compared the pros and cons of various classification methods such as SVM, LR, decision tree, and random forest algorithm.
      \end{itemize}
      }
    \item
      \ressubsingleline{IMDB sentiment analysis}{Personal project}{2020.10 – 2020.12}
      {\small
      \begin{itemize}
      
       \item I use stop-words to clean the data set and use wordcloud to visualize the keyword. 	
       \item I use word2vec to vectorize the text.
       \item I used the bi-lstm model to classify comments with different emotions, and trained the model, and the accuracy of the model reached 85\% in the test set. 
       \item I use docker, Tensorflow-serving, and streamlit to deploy the model on the web and complete the visualization of the model.
      \end{itemize}
      }
   
     \item
        \ressubsingleline{AIATSS}{Data analysis intern}{2020.04-2020.6}
        {\small
      \begin{itemize}
      \item I write SQL and python scripts to check the company's data.
        \item I use jira to monitor the progress of the workflow in real time, and draw a report on the progress of the test in the group through the Excel pivot table.
         \item I have a deeper understanding of the testing process and the ETL development proces
         
       \end{itemize}
       }
        \item
           \ressubsingleline{TCL Industrial  research lab }{Data mining intern}{2020.07-2020.09}
           {\small
      \begin{itemize}
      \item I participated in the formulation of the business indicators of the recommendation system CTR through discussions with members of the group, and conducted statistical analysis based on the above indicators.
         \item I use spark to complete data cleaning and verification of abnormal data.
          \item I participate in the recommendation strategy discussion within the group, and participate in the classification of large-scale feature data (Random Fourier features SVM), clustering (minitach kmeans), and participate in feature selection and feature crossover work.
           \item I participate in the daily maintenance of the crawler code in the group to enrich my own data mining experience.

            \end{itemize}

             }
       \item
           \ressubsingleline{iPinyou.inc}{Machine learning engineer}{2021.02-present}
           {\small
      \begin{itemize}
      \item I use Hive and Sqoop to synchronize data and clean database tables.
         \item I understand through business, divide and define positive and negative samples, and solve the problem of small sample training in actual projects.
          \item I use my business knowledge to supplement the missing values, and at the same time, I make an attempt at feature intersection and feature construction in feature engineering.
           \item  I use LR, xgboost, and catboost to classify the defined problems, optimize the model, and give feasible explanations for the important features of the model.
            \end{itemize}

             }
  \end{itemize}


















